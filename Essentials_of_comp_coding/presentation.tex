\documentclass{beamer}
\usepackage{graphicx}
\usepackage{amsmath}
\usepackage{caption}
\mode<presentation>{
\usetheme{Warsaw}
}
\title{Essentials of Competitive Coding}
\author{Suyash Agrawal}
\date{September 19, 2017}
\begin{document}
	\begin{frame}
	\titlepage	
	\end{frame}
	\section{Introduction}
		\subsection{Motivation and Background}
			\begin{frame}{What is it?}
				\uncover<+->{Here is what wikipedia says:\\}
				\uncover<+->{
				\begin{quote}
					Competitive programming is a mind sport usually held over the Internet or a local network,
					involving participants trying to program according to provided specifications.
				\end{quote}
				}
				\uncover<+->{Why is it useful?\\}
				\begin{itemize}
					\item<+-> You will become a better, more efficient, and less error-prone coder.
					\item<+-> Your technical programming interviews will be a piece of cake.
					\item<+-> You win prizes and trips to places around the world as you compete with and meet
					 other motivated and smart students.
				\end{itemize}
			\end{frame}
			\begin{frame}{Pre-requisites}
				Here are some pre requisites:
				\pause
				\begin{itemize}
					\item<+-> Language basic of either C++ or Java.
					\item<+-> Basic algorithmic skills.
					\item<+->[] \textit{And most importantly,}
					\item<+-> Perseverance
				\end{itemize}
			\end{frame}
			\begin{frame}{Mathematics}
				Some knowledge of mathematics is required too, like:
				\pause
				\begin{itemize}
					\item<+-> Modulo Arithmetic \textit{Basic Number Theory}
					\item<+-> Fast Exponentiation
					\item<+-> Complexity Analysis
					\item<+-> Sieve of Eratosthenes
					\item<+-> Basic Geometry
				\end{itemize}
				\uncover<+->{Don't worry even if you have no idea about these. You will easily pick these up
				as you practice along.}
			\end{frame}
			\begin{frame}{Problem Structure}
				Basic structure of the problem is as follows:
				\begin{itemize}
					\item<2->\textbf{Problem Statement}: Describes the problem and what output is to be generated.
					Usually in form of a long story from which you have to extract the essence.
					\item<3->\textbf{Input}: Describes the format of input. Read carefully as missing out
					any minor detail lands you in wrong answer zone.
					\item<4->\textbf{Output}: Describes the output format of the problem . Just like above,
					this one also should be read carefully.
					\item<5->\textbf{Contraints}: These can include constraints on input, time, memory, code size, etc.
					\item<6->\textbf{Time Limit}: See if your algorithm can work in this range.
					\item<7->\textbf{Memory Limit}: Usually not a deal breaker (unless you are doing something ghastly!).
				\end{itemize}
			\end{frame}
			\begin{frame}{Example I}
				\uncover<1->{\textbf{Problem}: Consider a currency system in which there are notes of seven denominations, namely, Rs. 1,
				Rs. 2, Rs. 5, Rs. 10, Rs. 50, Rs. 100. If the sum of Rs. N is input, write a program to computer
				smallest number of notes that will combine to give Rs. N.}\\
				\uncover<2->{\textbf{Input}: The first line contains an integer T, total number of testcases.
				Then follow T lines, each line contains an integer N.}\\
				\uncover<3->{\textbf{Output}: Display the smallest number of notes that will combine to give N.}\\
				\uncover<4->{\textbf{Constraints}: $1 \leq T \leq 1000 \quad 1 \leq N \leq 1000000$}\\
				% \framebreak
			\end{frame}
			\begin{frame}{Example II}
				\uncover<1->{\textbf{Example}:\\\textbf{Input}\\3\\1200\\500\\242\\
										\textbf{Output}\\12\\5\\7}\\
				\uncover<2->{\textit{Let us write a program to solve this!}}
			\end{frame}
		\subsection{Resources}
			\begin{frame}{Practice Sites}
				Here are a few practice site:
				\pause
				\begin{itemize}
					\item<+-> Codeforces
					\item<+-> Codechef
					\item<+-> Topcoder
					\item<+-> Hacker Rank
					\item<+-> Hacker Earth
				\end{itemize}
			\end{frame}
			\begin{frame}{Algorithm and DS}
				Here are a few learning resources:
				\pause
				\begin{itemize}
					\item<+-> Code Monk $\rightarrow$ \href{https://www.hackerearth.com/practice/codemonk/}{Link}
					\item<+-> Top Coder DataScience Tutorial $\rightarrow$ \href{https://www.topcoder.com/community/data-science/data-science-tutorials/}{Link}
					\item<+-> NPTEL Videos \textit{Naveen Garg} (Data structures and Algorithms)
					\item<+-> Contest Tutorials
				\end{itemize}
			\end{frame}
			\begin{frame}{Language Reference}
				\begin{itemize}
					\item<+-> Use either C++ or Java . Prefer C++
					\item<+-> C++ Tutorial $\rightarrow$ \href{http://www.learncpp.com/}{Link}
					\item<+-> STL Tutorial $\rightarrow$ \href{https://www.topcoder.com/community/data-science/data-science-tutorials/power-up-c-with-the-standard-template-library-part-1/}{Link}
				\end{itemize}
			\end{frame}
			\begin{frame}{Competitions}
				\begin{itemize}
					\item<+-> Codeforces Contests \textit{\{Bi Weekly\}}
					\item<+-> Codechef Long, Lunchtime  \textit{\{Monthly\}}
					\item<+-> ACM ICPC \textit{\{Most Prestigious Programming Contest\}}
					\item<+-> Google APAC
					\item<+-> Facebook Hackercup
				\end{itemize}
			\end{frame}
	\section{Essentials}
		\begin{frame}{Algorithm Class}
			Basic class of algorithms:\textit{\{Increasing order of difficulty\}}
			\pause
			\begin{itemize}
				\item<+-> Ad-hoc
				\item<+-> Greedy
				\item<+-> Divide and Conquer
				\item<+-> Dynamic Programming
				\item<+-> Network Flows
			\end{itemize}
		\end{frame}
		\begin{frame}{Tools Needed}
			These things are sufficient to practice coding:	
			\pause
			\begin{itemize}
			\item<+-> Simple Text Editor. \textbf{No IDE !}
			\item<+-> Compiler. GCC is standard
			\item<+-> Preferably Linux OS (Ubuntu works best)
			\item<+-> Internet Connection (for problem statements)
			\item<+-> Online compilers like \href{https://www.ideone.com/}{Ideone} ,
						\href{https://www.codechef.com/ide}{Codechef IDE} can also be used.
						\textit{Beware, they are painfully slow}
			\end{itemize}
		\end{frame}
	\section{Example Problems}
		\begin{frame}
			Let us look at some basic problems.
		\end{frame}
		\begin{frame}{Fibonacci Numbers}
			Write a program to return $n^{th}$ Fibonacci number.\\
			Fibonacci numbers follow the series: $0,1,1,2,3,5,8,13,21,\dots$\\
			What is the complexity of your code ?
			\pause
			\uncover<+->{Exponential ?}\\
			\uncover<+->{If yes, try making it linear\\}

			\uncover<+->{\textbf{Hint}}\\
			\uncover<+->{\textit{Try writing a recurrence relation}}\\
			\uncover<+->{$Fib(0) = 0$}\\
			\uncover<+->{$Fib(1) = 1$}\\
			\uncover<+->{$Fib(n) = Fib(n-1)+Fib(n-2)\quad,n\geq2$}\\
		\end{frame}
		\begin{frame}{Exponentiation}
			Write a program to calculate $a^b$ given a and b as inputs. Both are natural numbers and the result
			is expressible in $int$ range.\\
			\pause
			\uncover<+->{Did you calculate it as: $\underbrace{a*a*\ldots*a}_{b\text{-times}}$ }\\
			\uncover<+->{\textit{Can you do better?}}\\
			\uncover<+->{\textit{Of Course!}}\\
			\uncover<+->{\textbf{Use Fast Exponentiation!}}\\
		\end{frame}
		\begin{frame}{Interval Selection}
			\uncover<+->{This is a harder problem than the above ones.\\}
			\uncover<+->{You are given n activities with their start and finish times.
			Select the maximum number of activities that can be performed by a single person,
			assuming that a person can only work on a single activity at a time.\\}
			\uncover<+->{\textit{Try doing it on your own. If nothing comes, then Google! }}
		\end{frame}
	\section{Closing Note}
		\begin{frame}{Important Notes}
			\uncover<+->{Here are some general closing notes:}
			\begin{itemize}
				\item<+-> Practice Regularly.
				\item<+-> Make a group of enthusiastic people to participate in contests.
				\item<+-> Practice Regularly.
			\end{itemize}


		\end{frame}
		\begin{frame}
			\begin{center}
				\textbf{Thank You}
			\end{center}
		\end{frame}
\end{document}
